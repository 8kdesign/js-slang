\newcommand{\Rule}[2]{\genfrac{}{}{0.7pt}{}{{\setlength{\fboxrule}{0pt}\setlength{\fboxsep}{3mm}\fbox{$#1$}}}{{\setlength{\fboxrule}{0pt}\setlength{\fboxsep}{3mm}\fbox{$#2$}}}}

\newcommand{\TruE}{\textbf{\texttt{true}}}
\newcommand{\FalsE}{\textbf{\texttt{false}}}
\newcommand{\AndOp}{\texttt{\&\&}}
\newcommand{\OrOp}{\texttt{||}}
\newcommand{\ThenOp}{\texttt{?}}
\newcommand{\ElseOp}{\texttt{:}}
\newcommand{\Rc}{\texttt{\}}}
\newcommand{\Lc}{\texttt{\{}}
\newcommand{\Rp}{\texttt{)}}
\newcommand{\Lp}{\texttt{(}}
\newcommand{\Fun}{\textbf{\texttt{function}}}
\newcommand{\Let}{\textbf{\texttt{let}}}
\newcommand{\Return}{\textbf{\texttt{return}}}
\newcommand{\Const}{\textbf{\texttt{const}}}
\newcommand{\If}{\textbf{\texttt{if}}}
\newcommand{\Else}{\textbf{\texttt{else}}}
\newcommand{\Bool}{\texttt{boolean}}
\newcommand{\Number}{\texttt{number}}
\newcommand{\String}{\texttt{string}}
\newcommand{\Undefined}{\texttt{undefined}}
\newcommand{\Null}{\texttt{null}}
\newcommand{\Any}{\texttt{any}}
\newcommand{\Void}{\texttt{void}}
\newcommand{\Pred}{\textit{Pred}}
\newcommand{\type}{\textit{type}}
\newcommand{\polytype}{\textit{polytype}}
\newcommand{\predtype}{\textit{predtype}}
\newcommand{\ExtractPos}{\ensuremath{\textit{Extract}^+}}
\newcommand{\ExtractNeg}{\ensuremath{\textit{Extract}^-}}

\newtheorem{definition}{Definition}[section]

\section{Type System}  

In Source \S 4 Typed, the Source \S 4 syntax is expanded to include type syntax such as type annotations and type aliases.
This allows names to be explicitly typed, and for type checks to be performed.

Support for \texttt{typeof} operations is also added to Source \S 4 Typed.

\subsection{Type Environment}

In order to keep track of the type of names in a program, we define a
\emph{type environment}, denoted by $\Gamma$. More formally,
the partial function $\Gamma$ from names to types expresses a 
context, in which a name $x$ is associated with type $\Gamma(x)$. 

We define a relation $\Gamma[x \leftarrow t]\Gamma'$ on type environments 
$\Gamma$, names $x$, types $t$, and type environments $\Gamma'$,
which constructs a type environment that behaves like the 
given one, except that the type of $x$ is $t$. More formally, 
if $\Gamma[x \leftarrow t]\Gamma'$, then $\Gamma'(y)$ is $t$, 
if $y=x$ and $\Gamma(y)$ 
otherwise. Obviously, this uniquely identifies $\Gamma'$ for
a given $\Gamma$, $x$, and $t$, and thus the type environment extension
relation is functional in its first three arguments.

The set of names, on which a type environment
$\Gamma$ is defined, is called the domain of $\Gamma$, 
denoted by $\textit{dom}(\Gamma)$.

For each non-overloaded primitive operator, we add a binding to our initial
type environment $\Gamma_0$ as follows:

\begin{eqnarray*}
& &
       \emptyset[ -_2 \leftarrow  (\Number, \Number) \rightarrow \Number]\\
&& \hspace{2mm} [ * \leftarrow  (\Number, \Number) \rightarrow \Number]\\
&& \hspace{2mm} [ / \leftarrow  (\Number, \Number) \rightarrow \Number]\\
&& \hspace{2mm} [ \% \leftarrow (\Number, \Number) \rightarrow \Number]\\
&& \hspace{2mm} [ \&\& \leftarrow (\Bool, \texttt{T}) \rightarrow \Bool\ |\ \texttt{T}]\\
&& \hspace{2mm} [ || \leftarrow   (\Bool, \texttt{T}) \rightarrow \Bool\ |\ \texttt{T}]\\
&& \hspace{2mm} [ ! \leftarrow \Bool \rightarrow \Bool]\\
&& \hspace{2mm} [ -_1 \leftarrow \Number \rightarrow \Number]\\
&& \hspace{2mm} [ \texttt{typeof} \leftarrow \Any \rightarrow \String]\Gamma_{-2}
\end{eqnarray*}

The overloaded binary primitives (with the exception of \texttt{+}, the handling of which will be elaborated in \nameref{typing-rules})
are handled as follows:

\begin{eqnarray*}
 & &
      \Gamma_{-2}
                 [ \texttt{===} \leftarrow (\Any,\ \Any) \rightarrow \Bool] \\
&& \hspace{6mm}  [ \texttt{!==} \leftarrow (\Any,\ \Any) \rightarrow \Bool] \\
&& \hspace{6mm}  [ \texttt{>} \leftarrow (\String,\ \String) \rightarrow \Bool\ |\ (\Number,\ \Number) \rightarrow \Bool]\\
&& \hspace{6mm}  [ \texttt{>=} \leftarrow (\String,\ \String) \rightarrow \Bool\ |\ (\Number,\ \Number) \rightarrow \Bool]\\
&& \hspace{6mm}  [ \texttt{<} \leftarrow (\String,\ \String) \rightarrow \Bool\ |\ (\Number,\ \Number) \rightarrow \Bool] \\
&& \hspace{6mm}  [ \texttt{<=} \leftarrow (\String,\ \String) \rightarrow \Bool\ |\ (\Number,\ \Number) \rightarrow \Bool]\Gamma_{-1}
\end{eqnarray*}

The Source \S 4 standard library functions and constants have their types defined as follows:

\begin{tabular}[fragile]{lllllllll}
$\Gamma_{-1}$
& $[$ & \texttt{display}      & $\leftarrow$  & $\Any$ & & & $]$ \\
& $[$ & \texttt{error}      & $\leftarrow$  & $\Any$ & & & $]$ \\
& $[$ & \texttt{Infinity}      & $\leftarrow$  & $\Number$ & & & $]$ \\
& $[$ & \texttt{is\_boolean}   & $\leftarrow$  & $\Any$ & $\rightarrow$ & $\Bool$ & $]$ \\
& $[$ & \texttt{is\_function}  & $\leftarrow$  & $\Any$ & $\rightarrow$ & $\Bool$ & $]$ \\
& $[$ & \texttt{is\_number}    & $\leftarrow$  & $\Any$ & $\rightarrow$ & $\Bool$ & $]$ \\
& $[$ & \texttt{is\_string}    & $\leftarrow$  & $\Any$ & $\rightarrow$ & $\Bool$ & $]$ \\
& $[$ & \texttt{is\_undefined} & $\leftarrow$  & $\Any$ & $\rightarrow$ & $\Bool$ & $]$ \\
& $[$ & \texttt{math\_abs} & $\leftarrow$  & $\Number$ & $\rightarrow$ & $\Number$ & $]$ \\
& $[$ & \texttt{math\_acos} & $\leftarrow$  & $\Number$ & $\rightarrow$ & $\Number$ & $]$ \\
& $[$ & \texttt{math\_acosh} & $\leftarrow$  & $\Number$ & $\rightarrow$ & $\Number$ & $]$ \\
& $[$ & \texttt{math\_asin} & $\leftarrow$  & $\Number$ & $\rightarrow$ & $\Number$ & $]$ \\
& $[$ & \texttt{math\_asinh} & $\leftarrow$  & $\Number$ & $\rightarrow$ & $\Number$ & $]$ \\
& $[$ & \texttt{math\_atan} & $\leftarrow$  & $\Number$ & $\rightarrow$ & $\Number$ & $]$ \\
& $[$ & \texttt{math\_atan2} & $\leftarrow$  & $(\Number, \Number)$ & $\rightarrow$ & $\Number$ & $]$ \\
& $[$ & \texttt{math\_atanh} & $\leftarrow$  & $\Number$ & $\rightarrow$ & $\Number$ & $]$ \\
& $[$ & \texttt{math\_cbrt} & $\leftarrow$  & $\Number$ & $\rightarrow$ & $\Number$ & $]$ \\
& $[$ & \texttt{math\_ceil} & $\leftarrow$  & $\Number$ & $\rightarrow$ & $\Number$ & $]$ \\
& $[$ & \texttt{math\_clz32} & $\leftarrow$  & $\Number$ & $\rightarrow$ & $\Number$ & $]$ \\
& $[$ & \texttt{math\_cos} & $\leftarrow$  & $\Number$ & $\rightarrow$ & $\Number$ & $]$ \\
& $[$ & \texttt{math\_cosh} & $\leftarrow$  & $\Number$ & $\rightarrow$ & $\Number$ & $]$ \\
& $[$ & \texttt{math\_exp} & $\leftarrow$  & $\Number$ & $\rightarrow$ & $\Number$ & $]$ \\
& $[$ & \texttt{math\_expm1} & $\leftarrow$  & $\Number$ & $\rightarrow$ & $\Number$ & $]$ \\
& $[$ & \texttt{math\_floor} & $\leftarrow$  & $\Number$ & $\rightarrow$ & $\Number$ & $]$ \\
& $[$ & \texttt{math\_fround} & $\leftarrow$  & $\Number$ & $\rightarrow$ & $\Number$ & $]$ \\
& $[$ & \texttt{math\_hypot} & $\leftarrow$  & $\Any$ & & & $]$ \\
& $[$ & \texttt{math\_imul} & $\leftarrow$  & $(\Number, \Number)$ & $\rightarrow$ & $\Number$ & $]$ \\
& $[$ & \texttt{math\_LN2} & $\leftarrow$  & $\Number$ & & & $]$ \\
& $[$ & \texttt{math\_LN10} & $\leftarrow$  & $\Number$ & & & $]$ \\
& \end{tabular}

\begin{tabular}[fragile]{lllllllll}
& $[$ & \texttt{math\_log} & $\leftarrow$  & $\Number$ & $\rightarrow$ & $\Number$ & $]$ \\
& $[$ & \texttt{math\_log1p} & $\leftarrow$  & $\Number$ & $\rightarrow$ & $\Number$ & $]$ \\
& $[$ & \texttt{math\_log2} & $\leftarrow$  & $\Number$ & $\rightarrow$ & $\Number$ & $]$ \\
& $[$ & \texttt{math\_LOG2E} & $\leftarrow$  & $\Number$ & & & $]$ \\
& $[$ & \texttt{math\_log10} & $\leftarrow$  & $\Number$ & $\rightarrow$ & $\Number$ & $]$ \\
& $[$ & \texttt{math\_LOG10E} & $\leftarrow$  & $\Number$ & & & $]$ \\
& $[$ & \texttt{math\_max} & $\leftarrow$  & $\Any$ & & & $]$ \\
& $[$ & \texttt{math\_min} & $\leftarrow$  & $\Any$ & & & $]$ \\
& $[$ & \texttt{math\_PI} & $\leftarrow$  & $\Number$ & & & $]$ \\
& $[$ & \texttt{math\_pow} & $\leftarrow$  & $(\Number, \Number)$ & $\rightarrow$ & $\Number$ & $]$ \\
& $[$ & \texttt{math\_random} & $\leftarrow$  & $()$ & $\rightarrow$ & $\Number$ & $]$ \\
& $[$ & \texttt{math\_round} & $\leftarrow$  & $\Number$ & $\rightarrow$ & $\Number$ & $]$ \\
& $[$ & \texttt{math\_sign} & $\leftarrow$  & $\Number$ & $\rightarrow$ & $\Number$ & $]$ \\
& $[$ & \texttt{math\_sin} & $\leftarrow$  & $\Number$ & $\rightarrow$ & $\Number$ & $]$ \\
& $[$ & \texttt{math\_sinh} & $\leftarrow$  & $\Number$ & $\rightarrow$ & $\Number$ & $]$ \\
& $[$ & \texttt{math\_sqrt} & $\leftarrow$  & $\Number$ & $\rightarrow$ & $\Number$ & $]$ \\
& $[$ & \texttt{math\_SQRT1\_2} & $\leftarrow$  & $\Number$ & & & $]$ \\
& $[$ & \texttt{math\_SQRT2} & $\leftarrow$  & $\Number$ & & & $]$ \\
& $[$ & \texttt{math\_tan} & $\leftarrow$  & $\Number$ & $\rightarrow$ & $\Number$ & $]$ \\
& $[$ & \texttt{math\_tanh} & $\leftarrow$  & $\Number$ & $\rightarrow$ & $\Number$ & $]$ \\
& $[$ & \texttt{math\_trunc} & $\leftarrow$  & $\Number$ & $\rightarrow$ & $\Number$ & $]$ \\
& $[$ & \texttt{NaN} & $\leftarrow$  & $\Number$ & & & $]$ \\
& $[$ & \texttt{parse\_int} & $\leftarrow$  & $(\String, \Number)$ & $\rightarrow$ & $\Number$ & $]$ \\
& $[$ & \texttt{prompt} & $\leftarrow$  & $\String$ & $\rightarrow$ & $\String$ & $]$ \\
& $[$ & \texttt{get\_time} & $\leftarrow$  & $()$ & $\rightarrow$ & $\Number$ & $]$ \\
& $[$ & \texttt{stringify} & $\leftarrow$  & $\Any$ & $\rightarrow$ & $\String$ & $]$ \\
& $[$ & \texttt{undefined} & $\leftarrow$  & $\Undefined$ & & & $]$ \\
& $[$ & \texttt{null}      & $\leftarrow$  & $\Null$ & & & $]$ \\
& $[$ & \texttt{pair} & $\leftarrow$  & $(T_1,\ T_2)$ & $\rightarrow$ & \textit{Pair<} $T_1,\ T_2$ \textit{>} & $]$ \\
& $[$ & \texttt{head} & $\leftarrow$  & \textit{Pair<} $T_1,\ T_2$ \textit{>} & $\rightarrow$ & $T_1$ & $]$ \\
& $[$ & \texttt{tail} & $\leftarrow$  & \textit{Pair<} $T_1,\ T_2$ \textit{>} & $\rightarrow$ & $T_2$ & $]$ \\
& $[$ & \texttt{list} & $\leftarrow$  & $(T_1, \ldots, T_n)$ & $\rightarrow$ & \textit{List<} $T_1\ |\ \ldots\ |\ T_n$ \textit{>} & $]$ \\
& $[$ & \texttt{is\_pair} & $\leftarrow$  & $\Any$ & $\rightarrow$ & $\Bool$ & $]$ \\
& $[$ & \texttt{is\_null} & $\leftarrow$  & $\Any$ & $\rightarrow$ & $\Bool$ & $]$ \\
& $[$ & \texttt{is\_list} & $\leftarrow$  & $\Any$ & $\rightarrow$ & $\Bool$ & $]$ \\
& $[$ & \texttt{set\_head} & $\leftarrow$  & \textit{Pair<} $T_1,\ T_2$ \textit{>} & $\rightarrow$ & $\Undefined$ & $]$ \\
& $[$ & \texttt{set\_tail} & $\leftarrow$  & \textit{Pair<} $T_1,\ T_2$ \textit{>} & $\rightarrow$ & $\Undefined$ & $]$ \\
& $[$ & \texttt{is\_array} & $\leftarrow$  & $\Any$ & $\rightarrow$ & $\Bool$ & $]$ \\
& $[$ & \texttt{array\_length} & $\leftarrow$  & $\Any$[] & $\rightarrow$ & $\Number$ & $]$ \\
& $[$ & \texttt{apply\_in\_underlying\_javascript} & $\leftarrow$ & $(\Any,$ \textit{List<} $\Any$ \textit{>}$)$ & $\rightarrow$ & $\Any$ & $]$ \\
& $[$ & \texttt{tokenize} & $\leftarrow$  & $\String$ & $\rightarrow$ & \textit{List<} $\String$ \textit{>} & $]$ \\
& $[$ & \texttt{parse} & $\leftarrow$  & $\String$ & $\rightarrow$ & \textit{Program}\ |\ \textit{Statement} & $]$ $\Gamma_0$ \\
& \end{tabular}

For details on the types used by the parse function, refer to the appendix on parse tree types.

In order to support the definition of type aliases, we define a separate
\emph{type alias environment}, denoted by $\Gamma_{alias}$. Unlike $\Gamma$,
$\Gamma_{alias}$ binds names to special \emph{type functions} of the form $<T_1, \ldots, T_n> \rightarrow t$
where $T_1 \ldots T_n$ are type parameters $t$ is the return type expressed in terms of $T_1 \ldots T_n$.
$<>$ is used to differentiate type functions from function types, which are of the form $(t_1, \ldots, t_n) \rightarrow t$.

Since $\Gamma$ and $\Gamma_{alias}$ are separate environments, the same name $x$ can be used for both variables and type aliases.

The initial type alias environment $\Gamma_{alias0}$ is as follows:

\begin{tabular}[fragile]{lllllllll}
$\Gamma_{alias-1}$
& $[$ & \texttt{Pair} & $\leftarrow$ & $<T_{head}, T_{tail}>$ & $\rightarrow$ & $Pair<T_{head}, T_{tail}>$ & $]$ \\
& $[$ & \texttt{List} & $\leftarrow$ & $<T_{elem}>$ & $\rightarrow$ & $List<T_{elem}>$ & $]$ $\Gamma_{alias0}$ \\
& \end{tabular}

\subsection{Success Types}

In order for type checks to be performed in Source \S 4 Typed, we introduce the notion of success types.

We first define the special $\Any$ type:
\begin{definition}
$\Any$ is the union of all possible types.
\end{definition}

Success typing in Source Typed is defined as follows:

\begin{definition}
Type $t'$ is a success type of type $t$ if $\exists x (x \in t \wedge x \in t')$.
Alternatively: $t \wedge t' \neq \emptyset$.
\end{definition}

In Source Typed, type checks are performed by checking that the actual type is a success type of the expected type.
This means that type errors will be thrown if and only if a definite clash in types at runtime is detected.
Given that $\Any$ is the union of all possible types, this also means that the $\Any$ type is guaranteed not to produce any type errors.

\subsection{Typing Relations}
\label{typing-rules}

To perform type checking on the program, typing relations are applied to every statement and expression in the program.

Names that do not have a type declared will be assumed to have the $\Any$ type.

\subsubsection{Typing Relations on Expressions}

The derived type of primitive expressions is their literal type, which is an element of its corresponding basic type.

\noindent
\[
  \Rule{}{\Gamma, \Gamma_{alias} \vdash n : \textit{literal type}\ n}
  \quad
  \Rule{}{\Gamma, \Gamma_{alias} \vdash s : \textit{literal type}\ s}
\]
\noindent

where $n$ denotes any literal number and $s$ denotes any literal string.

\noindent
\[
  \Rule{}{\Gamma, \Gamma_{alias} \vdash \TruE : \textit{literal type}\ \TruE}
  \quad
  \Rule{}{\Gamma, \Gamma_{alias} \vdash \FalsE : \textit{literal type}\ \FalsE}
\]
\noindent

For names, the type must be derived from the type environment.

\noindent
\[
  \Rule{}{\Gamma, \Gamma_{alias} \vdash x : \Gamma(x)}
\]
\noindent

For function applications (including applications of binary and unary operators), the following two type rules are used, depending on the type of $E_0$.

\noindent
\[
\Rule{\Gamma, \Gamma_{alias} \vdash E_0 : (t_1, \ldots, t_n) \rightarrow t \quad \Gamma, \Gamma_{alias} \vdash E_1 : t'_1, \ldots,  \Gamma, \Gamma_{alias} \vdash E_n : t'_n
  \quad (\forall 1 \leq i \leq n)(t'_i \wedge t_i \neq \emptyset)}{\Gamma, \Gamma_{alias} \vdash E_0\ \Lp \ E_1, \ldots, E_n\ \Rp : t}
\]
\noindent
\[
  \Rule{\Gamma, \Gamma_{alias} \vdash E_0 : \Any \quad \Gamma, \Gamma_{alias} \vdash E_1 : t'_1, \ldots, \Gamma, \Gamma_{alias} \vdash E_n : t'_n
    }{\Gamma, \Gamma_{alias} \vdash E_0\ \Lp \ E_1, \ldots, E_n\ \Rp : \Any}
\]
\noindent

The type of the operator must be a function type with the right number of parameters,
and the type of every argument must be a success type of the corresponding parameter type of the function type.
If all of the conditions are met, the type of the function application is the same
as the return type of the function type that is the type of the operator.
If the type of the operator is $\Any$, the return type will be $\Any$.

Applications of binary and unary operators are treated the same as function applications, with the exception of the \texttt{+} operator.
We use the $\subseteq$ operator to indicate that a type is a subset of another type, as defined below:

\begin{itemize}
\item{A type is a subset of type $\Number$ if it is of type $\Number$, literal number type,
  or a union type containing any number of literal number types.}
\item{A type is a subset of type $\String$ if it is of type $\String$, literal string type,
  or a union type containing any number of literal string types.}
\end{itemize}

For the \texttt{+} operator, the following rules are applied, in order of priority:

\begin{enumerate}
\item{If the expression on the left side is a subset of type $\Number$,
  check that the other expression is a success type of $\Number$. The return type is $\Number$.}
\item{If the expression on the left side is a subset of type $\String$,
  check that the other expression is a success type of $\String$. The return type is $\String$.}
\item{If the expression on the right side is a subset of type $\Number$,
  check that the other expression is a success type of $\Number$. The return type is $\Number$.}
\item{If the expression on the right side is a subset of type $\String$,
  check that the other expression is a success type of $\String$. The return type is $\String$.}
\item{If the expression on the left side cannot be narrowed to a subset of either $\Number$ or $\String$, check that both sides are success types of 
  $\Number\ |\ \String$. The return type is $\Number\ |\ \String$.}
\end{enumerate}

\noindent
\[
  \Rule{\Gamma, \Gamma_{alias} \vdash E_0 : t_0 \quad \Gamma, \Gamma_{alias} \vdash E_1 : t_1 \quad t_0 \subseteq \Number
    \quad t_1 \wedge \Number \neq \emptyset}{\Gamma, \Gamma_{alias} \vdash E_0\ + \ E_1 : \Number}
\]
\noindent
\[
  \Rule{\Gamma, \Gamma_{alias} \vdash E_0 : t_0 \quad \Gamma, \Gamma_{alias} \vdash E_1 : t_1 \quad t_0 \subseteq \String
    \quad t_1 \wedge \String \neq \emptyset}{\Gamma, \Gamma_{alias} \vdash E_0\ + \ E_1 : \String}
\]
\noindent
\[
  \Rule{\Gamma, \Gamma_{alias} \vdash E_0 : t_0 \quad \Gamma, \Gamma_{alias} \vdash E_1 : t_1 \quad t_1 \subseteq \Number
    \quad t_0 \wedge \Number \neq \emptyset}{\Gamma, \Gamma_{alias} \vdash E_0\ + \ E_1 : \Number}
\]
\noindent
\[
  \Rule{\Gamma, \Gamma_{alias} \vdash E_0 : t_0 \quad \Gamma, \Gamma_{alias} \vdash E_1 : t_1 \quad t_1 \subseteq \String
    \quad t_0 \wedge \String \neq \emptyset}{\Gamma, \Gamma_{alias} \vdash E_0\ + \ E_1 : \String}
\]
\noindent
\[
  \Rule{\Gamma, \Gamma_{alias} \vdash E_0 : t_0 \quad \Gamma, \Gamma_{alias} \vdash E_1 : t_1 \quad t_0 \wedge (\Number\ |\ \String) \neq \emptyset
    \quad t_1 \wedge (\Number\ |\ \String) \neq \emptyset}{\Gamma, \Gamma_{alias} \vdash E_0\ + \ E_1 : \Number\ |\ \String}
\]
\noindent

For lambda expressions, we temporarily extend $\Gamma$ with the declared types of all the function parameters,
and check the type of the function body against the declared return type.
As type syntax is optional, if type annotations are absent for any of the arguments or the return type, the type is assumed to be $\Any$.
The type of the lambda expression is then the function type with the declared types of the parameters and the return type. 

\noindent
\[
  \Rule{\Gamma [x_1 \leftarrow t_1] \cdots [x_n \leftarrow t_n] \vdash S : t' \quad t' \wedge t \neq \emptyset}{
    \Gamma, \Gamma_{alias} \vdash (x_1:\ t_1, \ldots, x_n:\ t_n) :\ t\ \texttt{=>}\ S : (t_1, \ldots, t_n) \rightarrow t}  
\]
\noindent

The type of a conditional expression is the union of the type of its consequent expression and its alternate expression.
The predicate expression of a conditional expression must be a success type of a boolean.

\noindent
\[
  \Rule{\Gamma, \Gamma_{alias} \vdash E_{pred} : t_{pred} \quad \Gamma, \Gamma_{alias} \vdash E_{cons} : t_{cons} \quad \Gamma, \Gamma_{alias} \vdash E_{alt} : t_{alt}
    \quad t_{pred} \wedge \Bool \neq \emptyset}{\Gamma, \Gamma_{alias} \vdash E_{pred}\ \texttt{?}\ E_{cons} : E_{alt} : t_{cons}\ |\ t_{alt}}
\]
\noindent

For as expressions, the type to cast the expression to must be a success type of the type of the expression.

\noindent
\[
  \Rule{\Gamma, \Gamma_{alias} \vdash E : t' \quad t \wedge t' \neq \emptyset}{\Gamma, \Gamma_{alias} \vdash E\ \texttt{as}\ t : t}  
\]
\noindent

\subsubsection{Typing Relations on Assignments}

For assignments, the type of the right expression must be a success type of the type of the left expression.
The type of the assignment itself is the type of the right expression.

\noindent
\[
  \Rule{\Gamma, \Gamma_{alias} \vdash E_0 : t_0 \quad \Gamma, \Gamma_{alias} \vdash E_1 : t_1 \quad t_1 \wedge t_0 \neq \emptyset}{
    \Gamma, \Gamma_{alias} \vdash E_0 = E_1 : t_1}
\]
\noindent

\subsubsection{Typing Relations on Statements}

Sequences in the top level are handled using the following steps:

\begin{enumerate}
\item{Type alias declarations are evaluated, which adds type aliases to $\Gamma_{alias}$ to construct $\Gamma'_{alias}$.}
\item{The declared types of constant/variable declarations are added to $\Gamma$ to construct $\Gamma'$.
  Note that the declaration statements themselves are yet to be checked.}
\item{All statements are checked using $\Gamma'$ and $\Gamma'_{alias}$.}
\item{The type of the sequence is the type of the last value-producing statement.}
\end{enumerate}

In the below rule, $D_n$ denotes constant/variable declarations of the form $\texttt{const/let}\ x_n \texttt{:}\ t_n = E_n\texttt{;}$.
If the type annotation for $x_n$ is absent, the declared type $t_n$ is assumed to be $\Any$.

\[
  \Rule{\begin{minipage}{140mm}
    $\Gamma_{alias} \vdash A_1 : \Gamma_{alias1}, \ldots, \Gamma_{aliasm-1} \vdash A_m : \Gamma'_{alias} \quad
    \Gamma [x_1 \leftarrow t_1] \cdots [x_n \leftarrow t_n] \Gamma'$ \\
    $\Gamma', \Gamma'_{alias} \vdash D_1 : t_1, \ldots, \Gamma', \Gamma'_{alias} \vdash D_n : t_n \quad
    \Gamma', \Gamma'_{alias} \vdash S_1 : t'_1, \ldots, \Gamma', \Gamma'_{alias} \vdash S_p : t'_p$\\
    \end{minipage}}{
    \Gamma, \Gamma_{alias} \vdash \{ A_1, \ldots, A_m, D_1, \ldots, D_n, S_1, \ldots, S_p \} : t'_p, \Gamma', \Gamma'_{alias}}
\]

For type alias declarations, the declared type $t$ for type alias name $T$ is first checked against the type environments.
Any type parameters declared are temporarily added to the type alias environment when checking the type of $t$
to ensure that the type parameters are only used within $t$ itself.
Then, the binding of $T$ to type function $<T_1, \ldots, T_n> \rightarrow t$ is added to the type alias environment.
If no type parameters are given, the type function is assumed to take in 0 type arguments.

\noindent
\[
  \Rule{\Gamma, \Gamma_{alias} \vdash t : t \quad \Gamma_{alias} [T \leftarrow <> \rightarrow t] \Gamma'_{alias}
    }{\Gamma, \Gamma_{alias} \vdash \texttt{type}\ T = t \texttt{;} : \Undefined, \Gamma'_{alias}}
\]
\noindent
\[
  \Rule{\Gamma, \Gamma_{alias}[T_1 \leftarrow T_1]\ldots[T_n \leftarrow T_n] \vdash t : t \quad
    \Gamma_{alias} [T \leftarrow <T_1, \ldots, T_n> \rightarrow t] \Gamma'_{alias}
    }{\Gamma, \Gamma_{alias} \vdash \texttt{type}\ T<T_1, \ldots, T_n> = t \texttt{;} : \Undefined, \Gamma'_{alias}}
\]
\noindent

For constant and variable declarations, the declared type $t$ is retrieved from the type environment.
If the declared type is a type reference to a type alias with name $T$,
$t$ is obtained by applying the type arguments $t_1, \ldots, t_n$ to the type function for $T$,
replacing all instances of type variables $T_1, \ldots, T_n$ in $t$ with $t_1, \ldots, t_n$ respectively.

The derived type of the expression $E$, $t_E$, must be a success type of $t$.
The type of the statement itself is $\Undefined$.

\noindent
\[
  \Rule{\Gamma \vdash E : t_E \quad t_E \wedge t \neq \emptyset}{
    \Gamma, \Gamma_{alias} \vdash \texttt{const/let}\ x \texttt{:}\ t = E\texttt{;} : \Undefined}
\]
\noindent
\[
  \Rule{\Gamma_{alias}(T)<t_1, \ldots, t_n> = t \quad \Gamma \vdash E : t_E \quad t_E \wedge t \neq \emptyset
    }{\Gamma, \Gamma_{alias} \vdash \texttt{const/let}\ x \texttt{:}\ T<t_1, \ldots, t_n> = E\texttt{;} : \Undefined}
\]
\noindent

The type of return statements and expression statements is the type of the expression in the statement.

\noindent
\[
  \Rule{\Gamma, \Gamma_{alias} \vdash E : t}{\Gamma, \Gamma_{alias} \vdash \texttt{return}\ E\texttt{;} : t}
  \quad
  \Rule{\Gamma, \Gamma_{alias} \vdash E : t}{\Gamma, \Gamma_{alias} \vdash E\texttt{;} : t}
\]
\noindent

The type of break and continue statements is $\Undefined$.

\noindent
\[
  \Rule{}{\Gamma, \Gamma_{alias} \vdash \texttt{break} \texttt{;} : \Undefined}
  \quad
  \Rule{}{\Gamma, \Gamma_{alias} \vdash \texttt{continue} \texttt{;} : \Undefined}
\]
\noindent

For blocks, $\Gamma$ is first extended temporarily to include the types of names declared in the block.
Then, the component statements are checked against the extended type environment.

For function body blocks and if statement blocks, we assume that whenever there is a return statement
or a conditional statement with a return statement within a block, it is the last statement in the block.
(One could consider a ``dead code'' error otherwise.)

The type of a function body or if statement block is the type of the return statement in the block.
If the block does not contain any return statements, the type is $\Void$,
which is a special type that is used to denote the return type of a function that does not return anything,
and changes to $\Undefined$ if unioned with another type that is not $\Void$.

In the below rule, $D_n$ denotes constant/variable declarations of the form $\texttt{const/let}\ x_n \texttt{:}\ t_n = E_n\texttt{;}$.
If the type annotation for $x_n$ is absent, the declared type $t_n$ is assumed to be $\Any$.

\noindent
\[
  \Rule{\Gamma [x_1 \leftarrow t_1] \cdots [x_m \leftarrow t_m] \Gamma_{temp} \quad
    \Gamma_{temp} \vdash D_1 : t_1, \ldots, \Gamma_{temp} \vdash D_m : t_m \quad
    \Gamma_{temp} \vdash S_1 : t'_1, \ldots, \Gamma_{temp} \vdash S_n : t'_n
    }{\Gamma, \Gamma_{alias} \vdash \{ D_1, \ldots, D_m, S_1, \ldots, S_n\} : 
      \begin{cases}
        t'_n & S_n\ \textit{is a return statement} \\
        \Void & S_n\ \textit{is not a return statement}
      \end{cases}
    }
\]
\noindent

The type of a block that is not a function body or if statement block is the type of last value-producing statement in the block.

In the below rule, $D_n$ denotes constant/variable declarations of the form $\texttt{const/let}\ x_n \texttt{:}\ t_n = E_n\texttt{;}$.
If the type annotation for $x_n$ is absent, the declared type $t_n$ is assumed to be $\Any$.
We also assume that $S_k$ is the last value-producing statement in the block.

\noindent
\[
  \Rule{\Gamma [x_1 \leftarrow t_1] \cdots [x_m \leftarrow t_m] \Gamma_{temp} \quad
    \Gamma_{temp} \vdash D_1 : t_1, \ldots, \Gamma_{temp} \vdash D_m : t_m \quad
    \Gamma_{temp} \vdash S_1 : t'_1, \ldots, \Gamma_{temp} \vdash S_n : t'_n
    }{\Gamma, \Gamma_{alias} \vdash \{ D_1, \ldots, D_m, S_1, \ldots, S_n\} : t'_k}
\]
\noindent

The type of a conditional statement or if statement is the union of the type of its consequent statement and its alternate statement.
The predicate expression of a conditional statement must be a success type of a boolean.

\noindent
\[
  \Rule{\Gamma, \Gamma_{alias} \vdash S_{pred} : t_{pred} \quad \Gamma, \Gamma_{alias} \vdash S_{cons} : t_{cons} \quad \Gamma, \Gamma_{alias} \vdash S_{alt} : t_{alt}
    \quad t_{pred} \wedge \Bool \neq \emptyset}{\Gamma, \Gamma_{alias} \vdash \texttt{if}\ (S_{pred})\ S_{cons}\
    \texttt{else}\ S_{alt} : t_{cons}\ |\ t_{alt}}
\]
\noindent

For \texttt{while} statements, the predicate expression must be a success type of $\Bool$. The rest of the statement is treated as a block.

\noindent
\[
  \Rule{\Gamma, \Gamma_{alias} \vdash E_{pred} : t_{pred} \quad \Gamma, \Gamma_{alias} \vdash S : t \quad t_{pred} \wedge \Bool \neq \emptyset}{
    \Gamma, \Gamma_{alias} \vdash \texttt{while}\ (\ E_{pred}\ )\ S : t}
\]
\noindent

For \texttt{for} statements, the typing is the same as that of a \texttt{while} statement,
except with additional type checks for the initialization and assignment expressions.

\noindent
\[
  \Rule{\begin{minipage}{140mm}
    $\Gamma, \Gamma_{alias} \vdash E_{init} : t_{init} \quad \Gamma, \Gamma_{alias} \vdash E_{pred} : t_{pred} \quad
    \Gamma, \Gamma_{alias} \vdash E_{assign} : t_{assign} \quad$ \\
    $\Gamma, \Gamma_{alias} \vdash S : t \quad t_{pred} \wedge \Bool \neq \emptyset$\\
    \end{minipage}}{
    \Gamma, \Gamma_{alias} \vdash \texttt{for}\ (\ E_{init} \texttt{;} E_{pred} \texttt{;} E_{assign}\ )\ S : t}
\]
\noindent
