\newcommand{\Rule}[2]{\genfrac{}{}{0.7pt}{}{{\setlength{\fboxrule}{0pt}\setlength{\fboxsep}{3mm}\fbox{$#1$}}}{{\setlength{\fboxrule}{0pt}\setlength{\fboxsep}{3mm}\fbox{$#2$}}}}

\newcommand{\TruE}{\textbf{\texttt{true}}}
\newcommand{\FalsE}{\textbf{\texttt{false}}}
\newcommand{\AndOp}{\texttt{\&\&}}
\newcommand{\OrOp}{\texttt{||}}
\newcommand{\ThenOp}{\texttt{?}}
\newcommand{\ElseOp}{\texttt{:}}
\newcommand{\Rc}{\texttt{\}}}
\newcommand{\Lc}{\texttt{\{}}
\newcommand{\Rp}{\texttt{)}}
\newcommand{\Lp}{\texttt{(}}
\newcommand{\Fun}{\textbf{\texttt{function}}}
\newcommand{\Let}{\textbf{\texttt{let}}}
\newcommand{\Return}{\textbf{\texttt{return}}}
\newcommand{\Const}{\textbf{\texttt{const}}}
\newcommand{\If}{\textbf{\texttt{if}}}
\newcommand{\Else}{\textbf{\texttt{else}}}
\newcommand{\Bool}{\texttt{boolean}}
\newcommand{\Number}{\texttt{number}}
\newcommand{\String}{\texttt{string}}
\newcommand{\Undefined}{\texttt{undefined}}
\newcommand{\Null}{\texttt{null}}
\newcommand{\Any}{\texttt{any}}
\newcommand{\Void}{\texttt{void}}
\newcommand{\Pred}{\textit{Pred}}
\newcommand{\type}{\textit{type}}
\newcommand{\polytype}{\textit{polytype}}
\newcommand{\predtype}{\textit{predtype}}
\newcommand{\ExtractPos}{\ensuremath{\textit{Extract}^+}}
\newcommand{\ExtractNeg}{\ensuremath{\textit{Extract}^-}}

\newtheorem{definition}{Definition}[section]

\section{Type System}  

In Source \S 3 Typed, the Source \S 3 syntax is expanded to include type syntax such as type annotations and type aliases.
This allows names to be explicitly typed, and for type checks to be performed.

Support for \texttt{typeof} operations is also added to Source \S 3 Typed.

\subsection{Type Environment}

In order to keep track of the type of names in a program, we define a
\emph{type environment}, denoted by $\Gamma$. More
formally,
the partial function $\Gamma$ from names to types expresses a 
context, in which a name $x$ is associated with type $\Gamma(x)$. 

We define a relation $\Gamma[x \leftarrow t]\Gamma'$ on type environments 
$\Gamma$, names $x$, types $t$, and type environments $\Gamma'$,
which constructs a type environment that behaves like the 
given one, except that the type of $x$ is $t$. More formally, 
if $\Gamma[x \leftarrow t]\Gamma'$, then $\Gamma'(y)$ is $t$, 
if $y=x$ and $\Gamma(y)$ 
otherwise. Obviously, this uniquely identifies $\Gamma'$ for
a given $\Gamma$, $x$, and $t$, and thus the type environment extension
relation is functional in its first three arguments.

The set of names, on which a type environment
$\Gamma$ is defined, is called the domain of $\Gamma$, 
denoted by $\textit{dom}(\Gamma)$.

For each non-overloaded primitive operator, we add a binding to our initial
type environment $\Gamma_0$ as follows:

\begin{eqnarray*}
& &
       \emptyset[ -_2 \leftarrow  (\Number, \Number) \rightarrow \Number]\\
&& \hspace{2mm} [ * \leftarrow  (\Number, \Number) \rightarrow \Number]\\
&& \hspace{2mm} [ / \leftarrow  (\Number, \Number) \rightarrow \Number]\\
&& \hspace{2mm} [ \% \leftarrow (\Number, \Number) \rightarrow \Number]\\
&& \hspace{2mm} [ \&\& \leftarrow (\Bool, \texttt{T}) \rightarrow \Bool\ |\ \texttt{T}]\\
&& \hspace{2mm} [ || \leftarrow   (\Bool, \texttt{T}) \rightarrow \Bool\ |\ \texttt{T}]\\
&& \hspace{2mm} [ ! \leftarrow \Bool \rightarrow \Bool]\\
&& \hspace{2mm} [ -_1 \leftarrow \Number \rightarrow \Number]\\
&& \hspace{2mm} [ \texttt{typeof} \leftarrow \Any \rightarrow \String]\Gamma_{-2}
\end{eqnarray*}

The overloaded binary primitives (with the exception of \texttt{+}, the handling of which will be elaborated in \nameref{typing-rules})
are handled as follows:

\begin{eqnarray*}
 & &
      \Gamma_{-2}
                 [ \texttt{===} \leftarrow (\Any,\ \Any) \rightarrow \Bool] \\
&& \hspace{6mm}  [ \texttt{!==} \leftarrow (\Any,\ \Any) \rightarrow \Bool] \\
&& \hspace{6mm}  [ \texttt{>} \leftarrow (\String,\ \String) \rightarrow \Bool\ |\ (\Number,\ \Number) \rightarrow \Bool]\\
&& \hspace{6mm}  [ \texttt{>=} \leftarrow (\String,\ \String) \rightarrow \Bool\ |\ (\Number,\ \Number) \rightarrow \Bool]\\
&& \hspace{6mm}  [ \texttt{<} \leftarrow (\String,\ \String) \rightarrow \Bool\ |\ (\Number,\ \Number) \rightarrow \Bool] \\
&& \hspace{6mm}  [ \texttt{<=} \leftarrow (\String,\ \String) \rightarrow \Bool\ |\ (\Number,\ \Number) \rightarrow \Bool]\Gamma_{-1}
\end{eqnarray*}

The Source \S 3 standard library functions and constants have their types defined as follows:

\begin{tabular}[fragile]{lllllllll}
$\Gamma_{-1}$
& $[$ & \texttt{display}      & $\leftarrow$  & $\Any$ & & & $]$ \\
& $[$ & \texttt{error}      & $\leftarrow$  & $\Any$ & & & $]$ \\
& $[$ & \texttt{Infinity}      & $\leftarrow$  & $\Number$ & & & $]$ \\
& $[$ & \texttt{is\_boolean}   & $\leftarrow$  & $\Any$ & $\rightarrow$ & $\Bool$ & $]$ \\
& $[$ & \texttt{is\_function}  & $\leftarrow$  & $\Any$ & $\rightarrow$ & $\Bool$ & $]$ \\
& $[$ & \texttt{is\_number}    & $\leftarrow$  & $\Any$ & $\rightarrow$ & $\Bool$ & $]$ \\
& $[$ & \texttt{is\_string}    & $\leftarrow$  & $\Any$ & $\rightarrow$ & $\Bool$ & $]$ \\
& $[$ & \texttt{is\_undefined} & $\leftarrow$  & $\Any$ & $\rightarrow$ & $\Bool$ & $]$ \\
& $[$ & \texttt{math\_abs} & $\leftarrow$  & $\Number$ & $\rightarrow$ & $\Number$ & $]$ \\
& $[$ & \texttt{math\_acos} & $\leftarrow$  & $\Number$ & $\rightarrow$ & $\Number$ & $]$ \\
& $[$ & \texttt{math\_acosh} & $\leftarrow$  & $\Number$ & $\rightarrow$ & $\Number$ & $]$ \\
& $[$ & \texttt{math\_asin} & $\leftarrow$  & $\Number$ & $\rightarrow$ & $\Number$ & $]$ \\
& $[$ & \texttt{math\_asinh} & $\leftarrow$  & $\Number$ & $\rightarrow$ & $\Number$ & $]$ \\
& $[$ & \texttt{math\_atan} & $\leftarrow$  & $\Number$ & $\rightarrow$ & $\Number$ & $]$ \\
& $[$ & \texttt{math\_atan2} & $\leftarrow$  & $(\Number, \Number)$ & $\rightarrow$ & $\Number$ & $]$ \\
& $[$ & \texttt{math\_atanh} & $\leftarrow$  & $\Number$ & $\rightarrow$ & $\Number$ & $]$ \\
& $[$ & \texttt{math\_cbrt} & $\leftarrow$  & $\Number$ & $\rightarrow$ & $\Number$ & $]$ \\
& $[$ & \texttt{math\_ceil} & $\leftarrow$  & $\Number$ & $\rightarrow$ & $\Number$ & $]$ \\
& $[$ & \texttt{math\_clz32} & $\leftarrow$  & $\Number$ & $\rightarrow$ & $\Number$ & $]$ \\
& $[$ & \texttt{math\_cos} & $\leftarrow$  & $\Number$ & $\rightarrow$ & $\Number$ & $]$ \\
& $[$ & \texttt{math\_cosh} & $\leftarrow$  & $\Number$ & $\rightarrow$ & $\Number$ & $]$ \\
& $[$ & \texttt{math\_exp} & $\leftarrow$  & $\Number$ & $\rightarrow$ & $\Number$ & $]$ \\
& $[$ & \texttt{math\_expm1} & $\leftarrow$  & $\Number$ & $\rightarrow$ & $\Number$ & $]$ \\
& $[$ & \texttt{math\_floor} & $\leftarrow$  & $\Number$ & $\rightarrow$ & $\Number$ & $]$ \\
& $[$ & \texttt{math\_fround} & $\leftarrow$  & $\Number$ & $\rightarrow$ & $\Number$ & $]$ \\
& $[$ & \texttt{math\_hypot} & $\leftarrow$  & $\Any$ & & & $]$ \\
& $[$ & \texttt{math\_imul} & $\leftarrow$  & $(\Number, \Number)$ & $\rightarrow$ & $\Number$ & $]$ \\
& $[$ & \texttt{math\_LN2} & $\leftarrow$  & $\Number$ & & & $]$ \\
& $[$ & \texttt{math\_LN10} & $\leftarrow$  & $\Number$ & & & $]$ \\
& $[$ & \texttt{math\_log} & $\leftarrow$  & $\Number$ & $\rightarrow$ & $\Number$ & $]$ \\
& $[$ & \texttt{math\_log1p} & $\leftarrow$  & $\Number$ & $\rightarrow$ & $\Number$ & $]$ \\
& $[$ & \texttt{math\_log2} & $\leftarrow$  & $\Number$ & $\rightarrow$ & $\Number$ & $]$ \\
& $[$ & \texttt{math\_LOG2E} & $\leftarrow$  & $\Number$ & & & $]$ \\
& $[$ & \texttt{math\_log10} & $\leftarrow$  & $\Number$ & $\rightarrow$ & $\Number$ & $]$ \\
& $[$ & \texttt{math\_LOG10E} & $\leftarrow$  & $\Number$ & & & $]$ \\
& $[$ & \texttt{math\_max} & $\leftarrow$  & $\Any$ & & & $]$ \\
& $[$ & \texttt{math\_min} & $\leftarrow$  & $\Any$ & & & $]$ \\
& $[$ & \texttt{math\_PI} & $\leftarrow$  & $\Number$ & & & $]$ \\
& $[$ & \texttt{math\_pow} & $\leftarrow$  & $(\Number, \Number)$ & $\rightarrow$ & $\Number$ & $]$ \\
& $[$ & \texttt{math\_random} & $\leftarrow$  & $()$ & $\rightarrow$ & $\Number$ & $]$ \\
& $[$ & \texttt{math\_round} & $\leftarrow$  & $\Number$ & $\rightarrow$ & $\Number$ & $]$ \\
& \end{tabular}

\begin{tabular}[fragile]{lllllllll}
& $[$ & \texttt{math\_sign} & $\leftarrow$  & $\Number$ & $\rightarrow$ & $\Number$ & $]$ \\
& $[$ & \texttt{math\_sin} & $\leftarrow$  & $\Number$ & $\rightarrow$ & $\Number$ & $]$ \\
& $[$ & \texttt{math\_sinh} & $\leftarrow$  & $\Number$ & $\rightarrow$ & $\Number$ & $]$ \\
& $[$ & \texttt{math\_sqrt} & $\leftarrow$  & $\Number$ & $\rightarrow$ & $\Number$ & $]$ \\
& $[$ & \texttt{math\_SQRT1\_2} & $\leftarrow$  & $\Number$ & & & $]$ \\
& $[$ & \texttt{math\_SQRT2} & $\leftarrow$  & $\Number$ & & & $]$ \\
& $[$ & \texttt{math\_tan} & $\leftarrow$  & $\Number$ & $\rightarrow$ & $\Number$ & $]$ \\
& $[$ & \texttt{math\_tanh} & $\leftarrow$  & $\Number$ & $\rightarrow$ & $\Number$ & $]$ \\
& $[$ & \texttt{math\_trunc} & $\leftarrow$  & $\Number$ & $\rightarrow$ & $\Number$ & $]$ \\
& $[$ & \texttt{NaN} & $\leftarrow$  & $\Number$ & & & $]$ \\
& $[$ & \texttt{parse\_int} & $\leftarrow$  & $(\String, \Number)$ & $\rightarrow$ & $\Number$ & $]$ \\
& $[$ & \texttt{prompt} & $\leftarrow$  & $\String$ & $\rightarrow$ & $\String$ & $]$ \\
& $[$ & \texttt{get\_time} & $\leftarrow$  & $()$ & $\rightarrow$ & $\Number$ & $]$ \\
& $[$ & \texttt{stringify} & $\leftarrow$  & $\Any$ & $\rightarrow$ & $\String$ & $]$ \\
& $[$ & \texttt{undefined} & $\leftarrow$  & $\Undefined$ & & & $]$ \\
& $[$ & \texttt{null}      & $\leftarrow$  & $\Null$ & & & $]$ \\
& $[$ & \texttt{pair} & $\leftarrow$  & $(T_1,\ T_2)$ & $\rightarrow$ & \textit{Pair<} $T_1,\ T_2$ \textit{>} & $]$ \\
& $[$ & \texttt{head} & $\leftarrow$  & \textit{Pair<} $T_1,\ T_2$ \textit{>} & $\rightarrow$ & $T_1$ & $]$ \\
& $[$ & \texttt{tail} & $\leftarrow$  & \textit{Pair<} $T_1,\ T_2$ \textit{>} & $\rightarrow$ & $T_2$ & $]$ \\
& $[$ & \texttt{list} & $\leftarrow$  & $(T_1, \ldots, T_n)$ & $\rightarrow$ & \textit{List<} $T_1\ |\ \ldots\ |\ T_n$ \textit{>} & $]$ \\
& $[$ & \texttt{is\_pair} & $\leftarrow$  & $\Any$ & $\rightarrow$ & $\Bool$ & $]$ \\
& $[$ & \texttt{is\_null} & $\leftarrow$  & $\Any$ & $\rightarrow$ & $\Bool$ & $]$ \\
& $[$ & \texttt{is\_list} & $\leftarrow$  & $\Any$ & $\rightarrow$ & $\Bool$ & $]$ \\
& $[$ & \texttt{set\_head} & $\leftarrow$  & \textit{Pair<} $T_1,\ T_2$ \textit{>} & $\rightarrow$ & $\Undefined$ & $]$ \\
& $[$ & \texttt{set\_tail} & $\leftarrow$  & \textit{Pair<} $T_1,\ T_2$ \textit{>} & $\rightarrow$ & $\Undefined$ & $]$ \\
& $[$ & \texttt{is\_array} & $\leftarrow$  & $\Any$ & $\rightarrow$ & $\Bool$ & $]$ \\
& $[$ & \texttt{array\_length} & $\leftarrow$  & $\Any$[] & $\rightarrow$ & $\Number$ & $]$ $\Gamma_0$ \\
& \end{tabular}

\subsection{Subtyping}

In order for type checks to be performed in Source \S 3 Typed, we introduce the notion of subtyping.

\begin{definition}
$t \subseteq t'$ is used to denote that type $t$ is a subtype of type $t'$.
\end{definition}

Subtyping is illustrated using the following type rules:

\noindent
\[
\Rule{}{t \subseteq \Any}
\qquad
\Rule{t'\ \textit{is a basic type} \quad t\ \textit{is a basic type} \quad t' = t}{t' \subseteq t}
\]
\noindent
\[
\Rule{t'\ \textit{is a literal type} \quad t\ \textit{is a basic type} \quad \texttt{typeof}\ t' = t}{t' \subseteq t}
\]
\noindent
\[
\Rule{(\forall 1 \leq i \leq n)(t_i \subseteq t'_i) \quad t' \subseteq t}{(t'_1, \ldots t'_n) \rightarrow t' \subseteq (t_1, \ldots t_n) \rightarrow t}
\qquad
\Rule{(\exists i)(t' \subseteq t_i)}{t'\ \subseteq\ t_1\ |\ \ldots\ |\ t_n}
\]
\noindent
\[
\Rule{t' = \textit{Pair<}\ t'_{head}, t'_{tail}\ \textit{>} \quad t = \textit{Pair<}\ t_{head}, t_{tail}\ \textit{>}
  \quad t'_{head} \subseteq t_{head} \quad t'_{tail} \subseteq t_{tail}}{t' \subseteq t}
\]
\noindent
\[
\Rule{t' = \textit{List<}\ t'_{elem}\ \textit{>} \quad t = \textit{List<}\ t_{elem}\ \textit{>}
  \quad t'_{elem} \subseteq t_{elem}}{t' \subseteq t}
\qquad
\Rule{t' = \Null \quad t = \textit{List<}\ t_{elem}\ \textit{>}}{t' \subseteq t}
\]
\noindent
\[
\Rule{t' = \textit{Pair<}\ t'_{head}, t'_{tail}\ \textit{>} \quad t = \textit{List<}\ t_{elem}\ \textit{>}
  \quad t'_{head} \subseteq t_{elem} \quad t'_{tail} \subseteq t}{t' \subseteq t}
\]
\noindent
\[
\Rule{t' = \textit{List<}\ t'_{elem}\ \textit{>} \quad t = \textit{Pair<}\ t_{head}, t_{tail}\ \textit{>}
  \quad t'_{elem} \subseteq t_{head} \quad t' \subseteq t_{tail}}{t' \subseteq t}
\]
\noindent
\[
\Rule{t' = t'_{elem}[] \quad t = t_{elem}[] \quad t'_{elem} \subseteq t_{elem}}{t' \subseteq t}
\]
\noindent

\subsection{Typing Relations}
\label{typing-rules}

To perform type checking on the program, typing relations are applied to every statement and expression in the program.

Note that a name that is declared to be of type $t$ will always refer to values of a type $t'$ such that $t' \subseteq t$ at runtime.
Names that do not have a type declared will be assumed to have the $any$ type; the $any$ type will not ever produce any type errors.

\subsubsection{Typing Relations on Expressions}

The derived type of primitive expressions is their literal type, which is always a subtype of its corresponding basic type.

\noindent
\[
  \Rule{}{\Gamma \vdash n : \textit{literal type}\ n}
  \quad
  \Rule{}{\Gamma \vdash s : \textit{literal type}\ s}
\]
\noindent

where $n$ denotes any literal number and $s$ denotes any literal string.

\noindent
\[
  \Rule{}{\Gamma \vdash \TruE : \textit{literal type}\ \TruE}
  \quad
  \Rule{}{\Gamma \vdash \FalsE : \textit{literal type}\ \FalsE}
\]
\noindent

For names, the type must be derived from the type environment.

\noindent
\[
  \Rule{}{\Gamma \vdash x : \Gamma(x)}
\]
\noindent

For function applications (including applications of binary and unary operators), the following two type rules are used, depending on the type of $E_0$.

\noindent
\[
\Rule{\Gamma \vdash E_0 : (t_1, \ldots, t_n) \rightarrow t \quad \Gamma \vdash E_1 : t'_1, \ldots,  \Gamma \vdash E_n : t'_n
  \quad (\forall 1 \leq i \leq n)(t'_i \subseteq t_i \vee t'_i = \Any)}{\Gamma \vdash (E_0\ \Lp \ E_1, \ldots, E_n\ \Rp) : t}
\]
\noindent
\[
  \Rule{\Gamma \vdash E_0 : \Any \quad \Gamma \vdash E_1 : t'_1, \ldots,  \Gamma \vdash E_n : t'_n
  \quad (\forall 1 \leq i \leq n)(t'_i \subseteq \Any)}{\Gamma \vdash (E_0\ \Lp \ E_1, \ldots, E_n\ \Rp) : \Any}
\]
\noindent

The type of the operator must be a function type with the right number of parameters,
and the type of every argument must be a subtype of the corresponding parameter type of the function type.
If all these conditions are met, the type of the function application is the same
as the return type of the function type that is the type of the operator.
If the type of the operator is $\Any$, the types for the arguments will be checked against $\Any$ and the return type will be $\Any$.

Applications of binary and unary operators are treated the same as function applications, with the exception of the \texttt{+} operator.
For the \texttt{+} operator, the following rules are applied, in order of priority:

\begin{itemize}
\item{If the expression on the left side is a subtype of type $\Number$,
  check that the other expression is a subtype of $\Number$. The return type is $\Number$.}
\item{If the expression on the left side is a subtype of type $\String$,
  check that the other expression is a subtype of $\String$. The return type is $\String$.}
\item{If the expression on the right side is a subtype of type $\Number$,
  check that the other expression is a subtype of $\Number$. The return type is $\Number$.}
\item{If the expression on the right side is a subtype of type $\String$,
  check that the other expression is a subtype of $\String$. The return type is $\String$.}
\item{If the expression on the left side cannot be narrowed to either $\Number$ or $\String$, check that both sides are subtypes of 
  $\Number\ |\ \String$. The return type is $\Any$.}
\end{itemize}

\noindent
\[
  \Rule{\Gamma \vdash E_0 : t_0 \quad \Gamma \vdash E_1 : t_1 \quad t_0 = \Number \vee t_0 = \textit{literal number type}
    \quad t_1 \subseteq \Number \vee t_1 = \Any}{\Gamma \vdash E_0\ + \ E_1 : \Number}
\]
\noindent
\[
  \Rule{\Gamma \vdash E_0 : t_0 \quad \Gamma \vdash E_1 : t_1 \quad t_0 = \String \vee t_0 = \textit{literal string type}
    \quad t_1 \subseteq \String \vee t_1 = \Any}{\Gamma \vdash E_0\ + \ E_1 : \String}
\]
\noindent
\[
  \Rule{\Gamma \vdash E_0 : t_0 \quad \Gamma \vdash E_1 : t_1 \quad t_1 = \Number \vee t_1 = \textit{literal number type}
    \quad t_0 \subseteq \Number \vee t_0 = \Any}{\Gamma \vdash E_0\ + \ E_1 : \Number}
\]
\noindent
\[
  \Rule{\Gamma \vdash E_0 : t_0 \quad \Gamma \vdash E_1 : t_1 \quad t_1 = \String \vee t_1 = \textit{literal string type}
    \quad t_0 \subseteq \String \vee t_0 = \Any}{\Gamma \vdash E_0\ + \ E_1 : \String}
\]
\noindent
\[
  \Rule{\Gamma \vdash E_0 : t_0 \quad \Gamma \vdash E_1 : t_1 \quad t_0 \subseteq (\Number\ |\ \String) \vee t_0 = \Any
    \quad t_1 \subseteq (\Number\ |\ \String) \vee t_1 = \Any}{\Gamma \vdash E_0\ + \ E_1 : \Any}
\]
\noindent

For lambda expressions, we extend $\Gamma$ with the declared types of all the function parameters,
and check the type of the function body against the declared return type.
As type syntax is optional, if type annotations are absent for any of the arguments or the return type, the type is assumed to be $\Any$.
The type of the lambda expression is then the function type with the declared types of the parameters and the return type. 

\noindent
\[
  \Rule{\Gamma [x_1 \leftarrow t_1] \cdots [x_n \leftarrow t_n] \Gamma' \quad \Gamma' \vdash S : t' \quad t' \subseteq t \vee t' = \Any}{
    \Gamma \vdash ( \Lp  x_1:\ t_1, \ldots, x_n:\ t_n \Rp :\ t\ \texttt{=>}\ \Lc\ S\ \Rc) : (t_1, \ldots, t_n) \rightarrow t}  
\]
\noindent

The type of a conditional expression is the union of the type of its consequent expression and its alternate expression.
The predicate expression of a conditional expression must be a subtype of a boolean.

\noindent
\[
  \Rule{\Gamma \vdash E_{pred} : t_{pred} \quad \Gamma \vdash E_{cons} : t_{cons} \quad \Gamma \vdash E_{alt} : t_{alt}
    \quad t_{pred} \subseteq \Bool \vee t_{pred} = \Any}{\Gamma \vdash (E_{pred}\ \texttt{?}\ E_{cons} : E_{alt}) : t_{cons}\ |\ t_{alt}}
\]
\noindent

The type of a literal array expression is the union of the types of all of the array's members.

\noindent
\[
  \Rule{\Gamma \vdash E_1 : t_1, \ldots, \Gamma \vdash E_n : t_n}{
    \Gamma \vdash [E_1, \ldots, E_n] : (t_1\ |\ \ldots\ |\ t_n)[]}
\]
\noindent

The type of an array access expression is the member type of the array. The array index must be a number.
If the type of the array being accessed is $\Any$, the type of the array access expression is $\Any$.

\noindent
\[
  \Rule{\Gamma \vdash E_0 : t_0[] \quad \Gamma \vdash E_1 : t_1 \quad t_1 \subseteq \Number \vee t_1 = \Any}{
    \Gamma \vdash E_0[E_1] : t_0}
  \quad
  \Rule{\Gamma \vdash E_0 : \Any \quad \Gamma \vdash E_1 : t_1 \quad t_1 \subseteq \Number \vee t_1 = \Any}{
    \Gamma \vdash E_0[E_1] : \Any}
\]
\noindent

For as expressions, the type to cast the expression to must be a subtype of the type of the expression.

\noindent
\[
  \Rule{\Gamma \vdash E : t' \quad t \subseteq t'}{\Gamma \vdash (E\ \texttt{as}\ t) : t}  
\]
\noindent

\subsubsection{Typing Relations on Assignments}

For assignments, the type of the right expression must be a subtype of the type of the left expression.
The type of the assignment itself is $\Undefined$.

\noindent
\[
  \Rule{\Gamma \vdash E_0 : t_0 \quad \Gamma \vdash E_1 : t_1 \quad t_1 \subseteq t_0 \vee t_1 = \Any}{
    \Gamma \vdash (E_0 = E_1) : \Undefined}
\]
\noindent

\subsubsection{Typing Relations on Statements}

For constant or variable declarations, the declared type of $x$, $t$, is added to the type environment.
As type syntax is optional, if the type annotation for $x$ absent, the declared type $t$ is assumed to be $\Any$.
The derived type of the expression $E$ must be a subtype of the type declared for name $x$.
The type of the constant declaration statement itself is $\Undefined$.

\noindent
\[
  \Rule{\Gamma [x \leftarrow t] \Gamma' \quad \Gamma' \vdash E : t' \quad t' \subseteq t \vee t' = \Any}{
    \Gamma \vdash (\texttt{const}\ x \texttt{:}\ t = E\texttt{;}) : \Undefined}
  \qquad
  \Rule{\Gamma [x \leftarrow t] \Gamma' \quad \Gamma' \vdash E : t' \quad t' \subseteq t \vee t' = \Any}{
    \Gamma \vdash (\texttt{let}\ x \texttt{:}\ t = E\texttt{;}) : \Undefined}
\]
\noindent

For type alias declarations, the declared type of $x$, $t$, is added to the type environment.
The type of the type alias declaration itself is $\Undefined$.

\noindent
\[
  \Rule{\Gamma [x \leftarrow t] \Gamma'}{\Gamma \vdash (\texttt{type}\ x = t\texttt{;}) : \Undefined}
\]
\noindent

The type of return statements and expression statements is the type of the expression in the statement.

\noindent
\[
  \Rule{\Gamma \vdash E : t}{\Gamma \vdash (\texttt{return}\ E\texttt{;}) : t}
  \quad
  \Rule{\Gamma \vdash E : t}{\Gamma \vdash (E\texttt{;}) : t}
\]
\noindent

The type of break and continue statements is $\Undefined$.

\noindent
\[
  \Rule{}{\Gamma \vdash (\texttt{break} \texttt{;}) : \Undefined}
  \quad
  \Rule{}{\Gamma \vdash (\texttt{continue} \texttt{;}) : \Undefined}
\]
\noindent

For blocks, $\Gamma$ is first extended to include the types of names declared in the block.
Then, the component statements are checked against the extended type environment.

We assume that whenever there is a return statement or a conditional statement with a return statement within a
block, it is the last statement in the block. (One could consider a ``dead code'' error otherwise.)

The type of a function body block is the type of the return statement in the block.
If the block does not contain any return statements, the type is $\Undefined$.

\noindent
\[
  \Rule{\Gamma [x_1 \leftarrow t_1] \cdots [x_n \leftarrow t_n] \Gamma' \quad
    \Gamma' \vdash S_1 : t'_1, \ldots, \Gamma' \vdash S_n : t'_n \quad S_n\ \textit{is a return statement}}{
    \Gamma \vdash \{ (\texttt{decl}\ x_1 \texttt{:}\ t_1 = E_1\texttt{;}) \ldots (\texttt{decl}\ x_1 \texttt{:}\ t_1 = E_1\texttt{;})\
    S_1, \ldots, S_n\} : t'_n}
\]
\noindent
\[
  \Rule{\Gamma [x_1 \leftarrow t_1] \cdots [x_n \leftarrow t_n] \Gamma' \quad
    \Gamma' \vdash S_1 : t'_1, \ldots, \Gamma' \vdash S_n : t'_n \quad S_n\ \textit{is not a return statement}}{
    \Gamma \vdash \{ (\texttt{decl}\ x_1 \texttt{:}\ t_1 = E_1 \texttt{;}) \ldots (\texttt{decl}\ x_1 \texttt{:}\ t_1 = E_1\texttt{;})\
    S_1, \ldots, S_n\} : \Undefined}
\]
\noindent

The type of a block outside of function bodies is the type of the last value-producing statement in the block.

\noindent
\[
  \Rule{\Gamma [x_1 \leftarrow t_1] \cdots [x_n \leftarrow t_n] \Gamma' \quad
    \Gamma' \vdash S_1 : t'_1, \ldots, \Gamma' \vdash S_n : t'_n \quad S_n\ \textit{is the last value-producing statement}}{
    \Gamma \vdash \{ (\texttt{decl}\ x_1 \texttt{:}\ t_1 = E_1\texttt{;}) \ldots (\texttt{decl}\ x_1 \texttt{:}\ t_1 = E_1\texttt{;})\
    S_1, \ldots, S_n\} : t'_n}
\]
\noindent

The type of a conditional statement or if statement is the union of the type of its consequent statement and its alternate statement.
The predicate expression of a conditional statement must be a subtype of a boolean.

\noindent
\[
  \Rule{\Gamma \vdash S_{pred} : t_{pred} \quad \Gamma \vdash S_{cons} : t_{cons} \quad \Gamma \vdash S_{alt} : t_{alt}
    \quad t_{pred} \subseteq \Bool \vee t_{pred} = \Any}{\Gamma \vdash (\texttt{if}\ (S_{pred})\ \{\ S_{cons}\ \}\
    \texttt{else}\ \{\ S_{alt}\ \}) : t_{cons}\ |\ t_{alt}}
\]
\noindent

For \texttt{while} statements, the predicate expression must be a subtype of $\Bool$. The rest of the statement is treated as a block.

\noindent
\[
  \Rule{\Gamma \vdash E_{pred} : t_{pred} \quad \Gamma \vdash S : t \quad t_{pred} \subseteq \Bool \vee t_{pred} = \Any }{
    \Gamma \vdash (\texttt{while}\ (\ E_{pred}\ )\ \{\ S\ \}) : t}
\]
\noindent

For \texttt{for} statements, the typing is the same as that of a \texttt{while} statement,
except with additional type checks for the initialization and assignment expressions.

\noindent
\[
  \Rule{\Gamma \vdash E_{init} : t_{init} \quad \Gamma \vdash E_{pred} : t_{pred} \quad \Gamma \vdash E_{assign} : t_{assign} \quad
    \Gamma \vdash S : t \quad t_{pred} \subseteq \Bool \vee t_{pred} = \Any }{
    \Gamma \vdash (\texttt{for}\ (\ E_{init} \texttt{;} E_{pred} \texttt{;} E_{assign}\ )\ \{\ S\ \}) : t}
\]
\noindent